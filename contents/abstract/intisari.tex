Bagian ini memuat uraian singkat (tidak lebih dari 250 kata) tetapi padat dan jelas serta
memberikan gambaran menyeluruh tentang isi tugas akhir. Intisari tugas akhir memuat apa dan mengapa
penelitian dikerjakan, bagaimana dikerjakan, dan apa hasil penting yang diperoleh dari penelitian.

Penelitian ini membahas tentang [APA: Topik utama penelitian]. Penelitian ini dilakukan karena [MENGAPA: Alasan urgensi atau latar belakang masalah]. Tujuan utama dari tugas akhir ini adalah untuk [MENGAPA: Tujuan spesifik]. Metode yang digunakan dalam penelitian ini adalah [BAGAIMANA: Pendekatan, algoritma, atau metodologi yang dipakai]. Proses pengerjaan meliputi tahapan [sebutkan secara singkat tahapan krusial]. Hasil penting yang diperoleh dari penelitian ini menunjukkan bahwa [HASIL: Temuan utama atau performa sistem]. Kesimpulan dari penelitian ini adalah [implikasi hasil].

Contoh:

Pada umumnya sistem perangkat lunak terdiri dari beberapa concern, premis dari masalah
ini adalah sebaran concern, di mana kebutuhan rancangan tertentu cenderung memotong grup inti fungsional modul. Teknik orientasi-objek yang menerapkan concern tersebut
cenderung menghasilkan kode yang tersebar, daya baca yang sulit, serta susah untuk
dikembangkan. Metodologi baru, aspect-oriented programming (AOP), memberikan fasilitas
modularisasi pemotong-lintasan/cross-cutting concern. Dengan menggunakan AOP, terdapat
cara untuk membuat penerapan sistem yang lebih mudah untuk dirancang, dipahami, dan
dipelihara. Lebih jauh lagi, AOP menjanjikan produktivitas yang lebih tinggi, peningkatan
kualitas, dan kemampuan lebih baik untuk menambahkan feature baru.
AspectJ adalah bahasa pemrograman yang digunakan secara luas untuk menerapkan
program-program berorientasi aspek di Java. Namun demikian, AspectJ masih belum memiliki
bahasa pemodelan yang dapat memenuhi perancangan program berorientasi aspek. Aspect
Oriented Design Model (AODM), sebagai sebuah model perancangan baru pada pengembangan
program dalam AspectJ, hanya memperluas konsep-konsep UML (Unified Modeling Language)
yang telah ada dengan menggunakan mekanisme perluasan UML untuk memberikan konsep
orientasi-aspek yang ada di dalam AspectJ. AODM menyediakan spesikasi model rancangan
orientasi-aspek untuk ditransformasikan menjadi model rancangan UML biasa.

\vspace{2ex} % Jarak sedikit
\noindent \textbf{Kata Kunci: Kata kunci 1, Kata kunci 2, Kata kunci 3, Kata kunci 4, Kata kunci 5}